\documentclass[paper=a4paper, fontsize=10pt, head_space=10mm, foot_space=17mm, gutter=17mm, line_length=185mm, twoside]{jlreq}
\usepackage{minted}
\usepackage{hyperref}
\pagestyle{plain}

\title{neko\_todo ソースリスト}
\author{美都}
\date{\today}

\begin{document}
\maketitle
\tableofcontents
\clearpage

\section{Rustソース}
\subsection{メインモジュール main.rs}
\inputminted[linenos, breaklines]{rust}{src-rs/main.rs}
\clearpage
\subsection{アプリケーションステータス app\_status.rs}
\inputminted[linenos, breaklines]{rust}{src-rs/app_status.rs}
\clearpage

\subsection{コンフィグ設定処理 config.rs}
\inputminted[linenos, breaklines]{rust}{src-rs/config.rs}
\clearpage

\subsection{アプリケーション設定情報の処理 setup.rs}
\inputminted[linenos, breaklines]{rust}{src-rs/setup.rs}
\clearpage

\subsection{todoモデル処理 todo.rs}
\inputminted[linenos, breaklines]{rust}{src-rs/todo.rs}
\clearpage

\subsection{データベースアクセス database.rs}
\inputminted[linenos, breaklines]{rust}{src-rs/database.rs}
\clearpage

\subsection{commandモジュール tauri::command関数群 command.rs}
\inputminted[linenos, breaklines]{rust}{src-rs/command.rs}
\clearpage

\subsubsection {todoリストの表示・編集 todo.rs}
\inputminted[linenos, breaklines]{rust}{src-rs/command/todo.rs}
\clearpage

\subsubsection {ユーザー操作関係 user.rs}
\inputminted[linenos, breaklines]{rust}{src-rs/command/user.rs}
\clearpage

\subsubsection {セッション操作関連 session.rs}
\inputminted[linenos, breaklines]{rust}{src-rs/command/session.rs}
\clearpage

\subsubsection {アプリケーションの状態操作 app\_state.rs}
\inputminted[linenos, breaklines]{rust}{src-rs/command/app_state.rs}
\clearpage

\section {フロントエンド React関係}
\subsection {index.html}
\inputminted[linenos, breaklines]{html}{index.html}
\clearpage

\subsection {メインCSSファイル}
\inputminted[linenos, breaklines]{css}{src-react/App.css}
\clearpage

\subsection {main.jsx}
\inputminted[linenos, breaklines]{jsx}{src-react/main.jsx}
\clearpage

\subsection {アプリケーションメイン App.jsx}
\inputminted[linenos, breaklines]{jsx}{src-react/App.jsx}
\clearpage

\subsection {全体のベースページ BasePage.jsx}
\inputminted[linenos, breaklines]{jsx}{src-react/BasePage.jsx}
\clearpage

\subsection {アプリケーションの初期化 Init.jsx}
\inputminted[linenos, breaklines]{jsx}{src-react/Init.jsx}
\clearpage

\subsection {ユーザー登録画面 RegistUser.jsx}
\inputminted[linenos, breaklines]{jsx}{src-react/RegistUser.jsx}
\clearpage

\subsection {ログイン画面 Login.jsx}
\inputminted[linenos, breaklines]{jsx}{src-react/Login.jsx}
\clearpage

\subsection {todoリストの表示 TodoList.jsx}
\inputminted[linenos, breaklines]{jsx}{src-react/TodoList.jsx}
\clearpage

\subsection {todoアイテム表示 TodoItem.jsx}
\inputminted[linenos, breaklines]{jsx}{src-react/TodoItem.jsx}
\clearpage

\subsection {todoリスト画面 ツールバー TodoListToolbar.jsx} 
\inputminted[linenos, breaklines]{jsx}{src-react/TodoListToolbar.jsx}
\clearpage

\subsection {todoアイテムの追加 AddTodo.jsx}
\inputminted[linenos, breaklines]{jsx}{src-react/AddTodo.jsx}
\clearpage

\subsection {todoアイテムの編集 EditTodo.jsx}
\inputminted[linenos, breaklines]{jsx}{src-react/EditTodo.jsx}
\clearpage

\subsection {todoアイテムの複製 PasteTodo.jsx}
\inputminted[linenos, breaklines]{jsx}{src-react/PasteTodo.jsx}
\clearpage

\subsection {todoアイテム内容の入力フォーム InputTodo.jsx}
\inputminted[linenos, breaklines]{jsx}{src-react/InputTodo.jsx}
\clearpage

\subsection {todoアイテム処理のための日付処理ユーティリティー str2date.jsx}
\inputminted[linenos, breaklines]{jsx}{src-react/str2date.jsx}
\clearpage

\section{データベース構成}
\subsection{テーブル生成スクリプト create\_table.sql}
\inputminted[linenos, breaklines]{sql}{mariadb/create_table.sql}
\clearpage

\end{document}
